% !TeX spellcheck = fr_FR
\title{%
	Mode d'emploi LoRa\\
	\large Un guide non exaustif de l'installation\\
	 compliquée d'une Gateway LoRa}
\date{27 avril 2023}
\author{Xavier Hueber\\Noé Lindelaub\\HE-ARC}
\fontsize{12}{12}
\documentclass{article}

%Margins
\usepackage{geometry}
\geometry{
	a4paper,
	left=19.1mm,
	right=19.1mm,
	top=25.4mm,
	bottom=25.4mm,
}

%Import packages
\usepackage{graphicx}
\usepackage{blindtext}
\usepackage[section]{placeins}
\usepackage{float}% If comment this, figure moves to Page 2
\usepackage{ragged2e}
\usepackage{framed}
\usepackage{mathtools} 
\usepackage{amsmath} 
\usepackage[citestyle=alphabetic,bibstyle=authortitle, backend=biber]{biblatex}
\bibliography{sources.bib}

\justifying

%Custom Macros
\newcommand{\XHframebox}[1] 
{
	\noindent\framebox{\noindent
		\begin{minipage}{\dimexpr\linewidth-2\fboxrule-2\fboxsep\relax}
			#1
		\end{minipage}
	}
}
%Change table of contents name
\renewcommand{\contentsname}{Table des matières}
%Removes Bibliography from Table of contents

\begin{document}
	\maketitle
	\newpage
	\tableofcontents
	\newpage
	\section{Introduction}
		Ce guide créé dans le cadre du cours de Systèmes Communicants 2 \textsuperscript\textcopyright, vous expliquera comment installer les logiciels nécessaires à la création d'une Gateway LoRa à partir d'un HAT Raspberry Pi Seeed disposant d'une puce RHF0M301.\\
		Nous utiliserons des nodes possédant d'un microprocesseur ESP32-S3 ainsi que d'un module série LoRa-E5-HF pour transmettre des messages par LoRaWAN\textsuperscript\textregistered.
		\subsection{Matériel utilisé}
			Dans le cadre de ce projet, nous avons utilisé:
			\begin{itemize}
				\item Un kit LoRa/LoRaWAN\textsuperscript\textregistered Gateway - 868MHz de Seeed.
				\item Un Raspberry Pi 3 modèle B.
				\item Deux ESP32-S3-DevKitC-1 v1.0 de Espressif.
				\item Deux modules LoRa-E5-HF de Seeed.
				\item Des antennes 868MHz SMA.
				\item Câble d'adaptation U-FL vers SMA.
				\item Une Gateway TheThingNetwork Kickstarter
				\item Une Gateway TheThingNetwork Indoor 
			\end{itemize}
		\subsection{Logiciels utilisés}
			Dans le cadre de ce projet, nous avons installé:
				\begin{itemize}
				\item Raspberry Pi OS 64-bit datant du 21 février 2023.
				\item ChirpStack v3.
				\item RHF0M301-ChirpStack.
				\item Un éditeur de texte moderne (ex: Sublime Text, VSCode, ...)
				\item Le compilateur IDF de Espressif
			\end{itemize}
	\newpage
	\section{Installation}
		\subsection{Installation du Raspberry Pi}
			Le Raspberry Pi nous servira 
		\subsection{Installation de l'environnement de développement Espressif}
			
	\newpage
	\section{Sources}
		\printbibliography
\end{document}