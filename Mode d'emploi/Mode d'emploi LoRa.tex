% !TeX spellcheck = fr_FR
\title{%
	Mode d'emploi LoRa\\
	\large Un guide non exaustif de l'installation\\
	 compliquée d'une Gateway LoRa}
\date{27 avril 2023}
\author{Xavier Hueber\\Noé Lindelaub\\HE-ARC}
\fontsize{12}{12}
\documentclass{article}

%Margins
\usepackage{geometry}
\geometry{
	a4paper,
	left=19.1mm,
	right=19.1mm,
	top=25.4mm,
	bottom=25.4mm,
}

%Import packages
\usepackage{graphicx}
\usepackage{blindtext}
\usepackage[section]{placeins}
\usepackage{float}% If comment this, figure moves to Page 2
\usepackage{ragged2e}
\usepackage{framed}
\usepackage{mathtools} 
\usepackage{amsmath}
\usepackage{subcaption}
\usepackage{hyperref}
\usepackage[citestyle=alphabetic,bibstyle=authortitle, backend=biber]{biblatex}
\bibliography{sources.bib}

\justifying

%Custom Macros
\newcommand{\XHframebox}[1] 
{
	\noindent\framebox{\noindent
		\begin{minipage}{\dimexpr\linewidth-2\fboxrule-2\fboxsep\relax}
			#1
		\end{minipage}
	}
}
%Change table of contents name
\renewcommand{\contentsname}{Table des matières}
%Removes Bibliography from Table of contents

\begin{document}
	\maketitle
	\newpage
	\tableofcontents
	\newpage
	\section{Introduction}
		Ce guide créé dans le cadre du cours de Systèmes Communicants 2\textsuperscript\textcopyright, vous expliquera comment installer les logiciels nécessaires à la création d'une Gateway LoRa à partir d'un HAT Raspberry Pi Seeed disposant d'une puce RHF0M301.\\
		Nous utiliserons des nodes possédant d'un microprocesseur ESP32-S3 ainsi que d'un module série LoRa-E5-HF pour transmettre des messages par LoRaWAN\textsuperscript\textregistered.
		\subsection{Matériel utilisé}
			Dans le cadre de ce projet, nous avons utilisé:
			\begin{itemize}
				\item Un kit LoRa/LoRaWAN\textsuperscript\textregistered Gateway - 868MHz de Seeed.
				\item Un Raspberry Pi 3 modèle B.
				\item Deux ESP32-S3-DevKitC-1 v1.0 de Espressif.
				\item Deux modules LoRa-E5-HF de Seeed.
				\item Des antennes 868MHz SMA.
				\item Câble d'adaptation U-FL vers SMA.
				\item Une Gateway TheThingNetwork Kickstarter
				\item Une Gateway TheThingNetwork Indoor 
			\end{itemize}
		\subsection{Logiciels utilisés}
			Dans le cadre de ce projet, nous avons installé:
				\begin{itemize}
				\item Raspberry Pi OS 64-bit datant du 21 février 2023.
				\item Raspberry Pi Imager v1.7.4
				\item ChirpStack v3.
				\item RHF0M301-ChirpStack.
				\item Un éditeur de code moderne (ex: Sublime Text, VSCode, ...)
				\item Le compilateur IDF de Espressif
			\end{itemize}
		\subsection{Glossaire}
			\begin{center}
			\begin{tabular}{ ||c|c|| } 
				\hline
				Terme & Acronyme \\ [0.5ex]
				\hline \hline
				TheThingNetwork & TTN \\ 
				Carte d'extension du Raspberry Pi & HAT \\
				\hline
			\end{tabular}
			\end{center}
	\newpage
	\section{Installation}
		\subsection{Installation du Raspberry Pi}
			\subsubsection{Raspberry Pi OS}
				\XHframebox{Le Raspberry Pi nous servira à recevoir, gérer et retransmettre les paquets reçu par LoRaWAN\textsuperscript\textregistered.}
				
				Nous avons commencé par installer Rapsberry Pi OS 64 bit sur une carte sd grâce au logiciel Raspberry Pi Imager (voir figure \ref{fig:raspberrypiimager}).\\
				Dans celui-ci, nous pouvons activer la connection SSH et configurer les logins de l'utilisateur, ainsi nous n'avons pas besoin d'utiliser un écran pour configurer le Raspberry Pi.
				\begin{figure}[H]
					\centering
					\begin{subfigure}{0.49\textwidth}
						\centering
						\includegraphics[width=\linewidth]{raspberrypi_imager}
						\caption{Menu}
						\label{fig:raspberrypiimager}
					\end{subfigure}
					\begin{subfigure}{0.49\textwidth}
						\centering
						\includegraphics[width=\linewidth]{raspberrypi_imager1}
						\caption{Configuration du SSH et de l'utilisateur}
						\label{fig:raspberrypiimager1}
					\end{subfigure}
					\caption{Raspberry Pi Imager}
				\end{figure}
			\subsubsection{Logiciels tiers}
				Maintenant que nous avons installé un système d'exploitation sur le Raspberry Pi, nous pouvons installer une application pour récupérer et gérer les paquets LoRaWAN\textsuperscript\textregistered.
				
				Nous voulions pour cela installer l'application proposé par Seeed pour l'utilisation de leur HAT, cependant nous n'avons pas trouvé de liens de téléchargement ni de guide d'installation disponible sur leur site web. 
				Nous avons alors commencé par chercher des applications tierces compatible avec le module RHF0M301. Une solution simple était d'installer basicstation, un logiciel développé par TTN qui transfère tous les paquets vers leurs serveurs.
				
				Cependant, nous ne voulions pas pour commencer des tests, être dépendant de TTN et de leurs services (ceux-ci étant fortement limités à 30 secondes d'émission par jour).
				Nous avons alors décidé de chercher une autre solution et sommes tombés sur ChirpStack. ChirpStack est un ensemble d'applications permettant la réception, l'envoi et la gestion de paquets LoRaWAN\textsuperscript\textregistered comme basicstation, mais elles permettent de gagner en flexibilité car chaque étape de traitement des paquets sont fait par une autre application de façon totalement indépendante.
				
				\paragraph{ChirpStack nous donne ainsi accès au système que vous pouvez voir dans la figure \ref{fig:systemechirpstack}:}
				\begin{enumerate}
					\item L'application Gateway viens communiquer avec le HAT du Raspberry Pi 3 et publie les informations reçues par MQTT.
					\item Le Network Server récupère alors les informations et les traite pour qu'elles soient utilisable par la User Application et publie sur MQTT.
					\item La ou les User(s) Application(s) souscrivent au Network Server et reçoivent alors les informations traités.
				\end{enumerate}

				\begin{figure}[H]
					\centering
					\includegraphics[width=0.7\linewidth]{Systeme_ChirpStack}
					\caption{Système applicatif ChirpStack}
					\label{fig:systemechirpstack}
				\end{figure}
				
		\subsection{Installation de l'environnement de développement Espressif}
			Maintenant que nous avons installé les tools requis pour faire fonctionner le Raspberry Pi, nous pouvons passer à l'installation de la toolchain ESP afin de pouvoir programmer nos deux Devkits. 
			\newline
			\newline
			Pour ce faire, nous nous rendons sur le site d'espressif et suivons le guide d'installation\footnote{\url{https://docs.espressif.com/projects/esp-idf/en/latest/esp32s3/get-started/index.html\#get-started-step-by-step}}. 
			\newline
			Nous avons ensuite compilé et flasher le projet Hello-World\footnote{Voir notre répertoire Github} pour tester le fonctionnement de la carte. C'est avec succès que celle-ci s'est mise en route et nous a donné ses premiers mots.
			
	\section{Programmation du Devkit ESP}
		Nous avons commencé par étudier la documentation fournie par Seeed pour son  module LoRa-E5-HF. Sachant qu'il se pilote par série, il nous a fallu comprendre comment utiliser le port série du Devkit ESP.
		
		Pour cela, nous avons trouvé l'exemple uart\_echo donné par Espressif qui nous a permis de comprendre comment configurer et utiliser l'uart.
		
		Nous avons ensuite importé un partie du code de uart\_echo dans notre application Lora\_send
	\newpage
	\section{Sources}
		\printbibliography
\end{document}